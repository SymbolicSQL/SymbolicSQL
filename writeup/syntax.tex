%!TEX root=writeup.tex
\section{Syntax}

Figure~\ref{tab:sql-syntax} shows the syntax of the DSL that we developed
in Rosette \cite{rosette}. It is a SQL like functional language. The 
translation from SQL to our DSL is straight forward. We support standard
Select-Projection-Join (SPJ) SQL queries. In addition, we also support 
aggregation and correlated subqueries. 

\begin{figure}[t]
\centering
\[
\begin{array}{llll}
  Query & ::=  &  \mathit{Table} & \text{(Input table)} \\
        & \; \mid & \textbf{SELECT} \quad \overline{Expr} \quad  Query      \\ 
        & \; \mid  & \textbf{PRODUCT} \quad Query \quad Query        \\
        & \; \mid & Query \quad \textbf{WHERE} \quad Pred        \\
        & \; \mid & Query \quad \textbf{UNION ALL} \quad Query   \\
        & \; \mid & \textbf{DISTINCT} \quad Query                \\
  Pred & ::= & Expr \quad \textbf{=} \quad Expr \\
       & \; \mid &  \textbf{NOT} \quad Pred      \\
       & \; \mid & \textbf{AND} \quad Pred \quad Pred      \\ 
       & \; \mid & \textbf{OR} \quad Pred \quad Pred \\
       & \; \mid & false \\
       & \; \mid & true  \\
       & \; \mid & \textbf{EXISTS} \quad SQL \\
  Expr & ::= & Column                     \\
        & \; \mid & Literal                  \\
        & \; \mid & function \quad Expr \ldots     \\
        & \; \mid & aggregator \quad Query   \\  
\end{array}
\]
\caption{SQL Syntax}
\label{tab:sql-syntax}
\end{figure}

There are minor differences between the syntax of our DSL and SQL. Instead of using \textbf{FROM} keyword, we use \textbf{PRODUCT} as 
a binary operator, and will be denoted to a cartesan product of 
two relations. For example, the user of language needs to write 
\textbf{(PRODUCT (PRODUCT a b) c)} instead of \textbf{FROM a, b, c}.  

Our DSL can express aggregation query (query with \textbf{GROUP BY}) by
using aggregation function over single-columned relation and correlated subqueries. This idea is firstly discovered by Buneman et. al \cite{comp_syntax}. For example, \textbf{SELECT a, sum(b) FROM R} 
is equivalent to \textbf{SELECT a, sum(SELECT b FROM R R1 WHERE R1.a = a ) FROM R}.

In practice, we found this syntax is easy to use in term of expressing 
rewriting rules. Translating existing rewriting rules from the database 
literatures and query optimizers requires very little efforts.

%%% Local Variables:
%%% mode: latex
%%% TeX-master: "writeup"
%%% End:
