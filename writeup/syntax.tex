%!TEX root=writeup.tex
\section{Syntax}

Figure~\ref{tab:sql-syntax} shows the syntax of $\mathsf{CoreSQL}$, the DSL that we developed for verification. It models a core fraction of SQL that is easy to reason about while remains very expressive: besides supporting standard
Select-Projection-Join (SPJ) SQL queries, we also support aggregation and correlated subqueries.  

\begin{figure}[t]
\centering
\[
\begin{array}{rcl}
  \mathit{Query} & ::=  &  \mathsf{Table} \\
        & \; \mid & \mathbf{SELECT}~\mathit{Exprs}\\
        && ~~~~\mathbf{FROM}~\mathit{Query}~\mathbf{WHERE}~\mathit{Pred}\\ 
        & \; \mid  & \mathbf{JOIN}~\mathit{Query}~\mathit{Query}        \\
        & \; \mid & \mathbf{SELECT\code{-}DISTINCT}~\mathit{Exprs}\\
        & & ~~~~\mathbf{FROM}~\mathit{Query}~\mathbf{WHERE}~\mathit{Pred}\\ 
  \mathit{Pred} & ::= & \mathit{Expr} ~ \mathbf{=}~\mathit{Expr} \\
       & \; \mid &  \mathbf{NOT} ~ \mathit{Pred}      \\
       & \; \mid & \mathbf{AND}~ \mathit{Pred} ~ \mathit{Pred}      \\ 
       & \; \mid & \mathbf{OR} ~ \mathit{Pred} ~ \mathit{Pred} \\
       & \; \mid & false \\
       & \; \mid & true  \\
       & \; \mid & \mathbf{EXISTS} ~ \mathit{Query} \\
  \mathit{Exprs} & ::= & \mathit{Expr}...                     \\
  \mathit{Expr} & ::= & \mathsf{column}                     \\
        & \; \mid & \mathsf{literal}                  \\
        & \; \mid & \mathsf{function}(\mathit{Expr}...)     \\
        & \; \mid & \mathbf{AGGR}(\mathsf{aggregator}, \mathit{Query})  \\  
\end{array}
\]
\caption{$\mathsf{CoreSQL}$ Syntax, where terminals $\mathsf{Table}$ stands for table reference,  $\mathsf{column}$ stands for column names from tables, $\mathsf{literal}$ stands for constants, $\mathsf{function}$ stands for $n$-nary operators that whose arguments are expressions and $\mathsf{aggregator}$ stands for aggregation functions.}
\label{tab:sql-syntax}
\end{figure}

There are several differences between the syntax of $\mathsf{CoreSQL}$ and SQL. Instead of using n-ary \textbf{JOIN...ON..} keyword, we use \textbf{JOIN} as a binary operator, and will be denoted to a cartesan product of two relations. For example, the user of language needs to write ``\lstinline[style=sql, keywords={}]{(JOIN (JOIN a b) c)}'' instead of ``\lstinline[style=sql, keywords={}]{FROM a, b, c}''.  

Our DSL can express aggregation query (query with ``\lstinline[style=sql, keywords={}]{GROUP BY}'') by
using aggregation function over single-columned relation and correlated subqueries. This idea is firstly discovered by Buneman et. al \cite{comp_syntax}. For example, query ``\lstinline[style=sql, keywords={}]{SELECT a, sum(b) FROM R}'' 
is equivalent to ``\lstinline[style=sql, keywords={}]{SELECT a, sum(SELECT b FROM R R1 WHERE R1.a = a) FROM R}''.


In practice, we find this syntax is easy to use in term of expressing 
rewriting rules and easy to reason about. Particularly, translating SQL queries to $\mathsf{CoreSQL}$ queries is mostly straight forward, and existing rewriting rules from the database 
literatures and query optimizers can be easily presented as two $\mathsf{CoreSQL}$ queries for us to reason about.

%%% Local Variables:
%%% mode: latex
%%% TeX-master: "writeup"
%%% End:
