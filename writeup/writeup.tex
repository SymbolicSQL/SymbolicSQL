\documentclass[10pt]{article}
\usepackage[pdftex]{graphicx}
\usepackage{wrapfig}
\usepackage{beramono}
\usepackage{enumerate}
\usepackage{hyperref}
\usepackage{fullpage}
\usepackage{cite}
\usepackage[top=1in, bottom=1in, left=0.85in, right=0.85in]{geometry}
\usepackage{bussproofs}
\usepackage{stmaryrd}
\usepackage{amsmath,amsthm,amsfonts,amssymb,amscd}
\usepackage{url}
\usepackage{listings}
\usepackage{stmaryrd}
\usepackage{inconsolata}
\usepackage{xspace}
\usepackage{adjustbox}
\usepackage{fancyhdr}
\usepackage{mathrsfs}
\usepackage{listings}
\usepackage{xcolor}
\usepackage{graphicx}

\usepackage{mathpartir}

\newcommand{\cL}{{\cal L}}
\newcommand{\sys}{\textsc{DopCert}\xspace}
\newcommand{\lbr}{\llbracket}
\newcommand{\rbr}{\rrbracket}
\newcommand{\gvd}{\Gamma \vdash}
\newcommand{\teq}{\triangleq}
\newcommand{\code}[1]{\texttt{\footnotesize #1}}



\newcommand{\flyingbox}[1]{\begin{flushleft}\fbox{{#1}}\end{flushleft}}
\newcommand{\concat}{\ensuremath{\!+\!\!\!\!+\!\,}}


\definecolor{forestgreen}{RGB}{64,139,64}
\lstdefinestyle{sql}{ % Define a style for your code snippet, multiple definitions can be made if, for example, you wish to insert multiple code snippets using different programming languages into one document
%backgroundcolor=\color{highlight}, % Set the background color for the snippet - useful for highlighting
language=sql,
basicstyle=\ttfamily, % The default font size and style of the code
breakatwhitespace=false, % If true, only allows line breaks at white space
breaklines=true, % Automatic line breaking (prevents code from protruding outside the box)
captionpos=b, % Sets the caption position: b for bottom; t for top
commentstyle=\color[rgb]{0,0.6,0}, % Style of comments within the code - dark green courier font
deletekeywords={VALUE, INPUT}, % If you want to delete any keywords from the current language separate them by commas
%escapeinside={\%}, % This allows you to escape to LaTeX using the character in the bracket
firstnumber=1, % Line numbers begin at line 1
frame=none, % Frame around the code box, value can be: none, leftline, topline, bottomline, lines, single, shadowbox
frameround=tttt, % Rounds the corners of the frame for the top left, top right, bottom left and bottom right positions
keywordstyle=\color{forestgreen},
morekeywords={PRODUCT}, % Add any functions no included by default here separated by commas
numbers=left, % Location of line numbers, can take the values of: none, left, right
numbersep=3pt, % Distance of line numbers from the code box
numberstyle=\large\color[rgb]{0.5,0.5,0.5}, % Style used for line numbers
rulecolor=\color{black}, % Frame border color
showstringspaces=false, % Don't put marks in string spaces
showtabs=false, % Display tabs in the code as lines
stepnumber=0, % The step distance between line numbers, i.e. how often will lines be numbered
tabsize=2, % Number of spaces per tab in the code
backgroundcolor=\color{white}
}
\lstset{escapeinside={@}{@}}

%for bussproofs abbreviation
\EnableBpAbbreviations

\newcommand\note[1]{\textcolor{red}{NOTE: #1}}

\begin{document}

\title{Bounded Verification of SQL Rewriting Rules}
\author{Chenglong Wang, Kaiyuan Zhang, Shumo Chu \\  
       Computer Science and Engineering\\ 
       University of Washington \\ 
       \{clwang, kaiyuanz, chushumo\}@cs.washington.edu  }
\date{}
\maketitle

%! TEX root=writeup.tex
\section{Introduction}
%!TEX root=writeup.tex
\section{Approach Overview}

In this section, we shall demonstrate our approach with an running example.
%!TEX root=writeup.tex
\section{Syntax}

%!TEX root=writeup.tex

\section{$\mathsf{CoreSQL}$ Denotation Semantics}

In this section, we introduce the denotation semantics of $\mathsf{CoreSQL}$, where each $\mathsf{CoreSQL}$ query is denoted into a Rosette program, and we verify the equivalence of two queries by verifying the their corresponding denoted Rosette program. 

\paragraph{Target Language} 

Before introducing denotation rules, we first introduce $\mathsf{CoreSQL}$ representations in Rosette. Concretely, we use the following Rosette constructs to represent queries.
\begin{itemize}\itemsep0pt
\item \emph{Table}: A table in Rosette is a value of type \code{list<pair<list<int>, int>}, and it is presented as a list of table rows, and each row is a pair, where the first element is a list of integers representing its content the the second element is a (non-negative) integer representing the repetitive number of the row in the table. For example, a table \code{(((1 2) . 2) ((2 5) . 3))} represents a 5 row table such that it contains two rows with content \code{(1 2)} and 3 rows with content \code{(2 5)}. Note that currently we only support integer tables and we can extend the row type to support more richer types including string, data etc.

\item \emph{Environment}: In order to support subqueries, we need to model the environment of a query (i.e. what are the outer level values available in current query). Traditional ways to model the environment is by adding bindings (variable names and their run-time values): the bindings are created by outer level queries and inner level queries will simply look up bindings by name to find values at runtime. However, HashMaps are not supported in Rosette as they are challenging to reason about, and this representation will make it very challenging to verify Rosette program equivalence. 

Instead, environment is modeled as a list of integers, and a outer level query will pass the list to inner level queries for value reference. As we will present later in denotation rules, we can pull up hash lookup into compile time and thus change runtime binding lookup to list reference, thus enable better verification performance.

\item \emph{Queries}: With the modeling of table and environment above, queries can then be represented as a Rosette function of type \code{Env -> Table}: it takes an environment as input (for outer-most queries, the environment is an empty list) and returns a table as output. Similarly, filters in $\mathsf{CoreSQL}$ queries are functions of type \code{env -> boolean} and values functions of type \code{env -> int} .
\end{itemize}

\paragraph{Denotation Rules} The denotation rules are presented in Figure~\ref{fig:denotation}. As mentioned above, each query is denoted into a Rosette function, and most denotation rules are straight forward. The most tricky one is the denotation rule for \code{Select} clause, and variable lookup: the former one requires us to compute variable bindings and the latter one requires us to correctly lookup value from the environment list.

To handle this, we add a compile-time environment $\Phi$ to store the bindings between variables and their locations in the \code{env} list. When we encounter a variable look up as in $\llbracket \mathit{c};\Phi \rrbracket $, we will first look up the variable in $\Phi$ to identify its position $i$ in the environment and then generate a list reference call \code{(lookup e i)} to enable runtime access. On the other hand, as presented in the denotation rule for \code{select} clause, the hash map is updated to make sure that relative position of each column name is consistent with its position in the environment. In this way, we shift hash lookup into compile time, and we only need to reason about list reference in compiled program.

\begin{figure}[ht]
\[
\begin{array}{rcl}
 \llbracket T; \Phi  \rrbracket& \leadsto& \lambda e.T\\
\llbracket \mathbf{join}(R_1,...,R_n);\Phi \rrbracket &\leadsto& \lambda e.(\llbracket R_1; \Phi\rrbracket~e)\times...\times(\llbracket R_n;\Phi\rrbracket~e)\\
\llbracket \mathbf{select}(V, R, f);\Phi \rrbracket &\leadsto& \lambda e. (\mathsf{map}~(\lambda x.\llbracket V;\mathtt{update}(\Phi, x, \mathtt{schema}(R)) \rrbracket~ e\concat x)\\
& & \qquad(\mathsf{filter}~(\lambda x.\llbracket f; \mathtt{update}(\Phi, x,\mathtt{schema}(R))\rrbracket~e\concat x)~ (\llbracket R;\Phi \rrbracket~e)))\\
\llbracket \mathbf{rename}(R, n, l);\Phi \rrbracket &\leadsto& \lambda e.(\llbracket R; \Phi\rrbracket~e)\\
\\
 \llbracket \mathbf{and}(f_1,...,f_n);\Phi \rrbracket &\leadsto& \lambda e.((\llbracket f_1;\Phi \rrbracket~e) \land ... \land (\llbracket f_n;\Phi \rrbracket~e))\\
\llbracket \mathbf{or}(f_1,...,f_n);\Phi \rrbracket &\leadsto& \lambda e.((\llbracket f_1;\Phi \rrbracket~e) \lor ... \lor (\llbracket f_n;\Phi \rrbracket~e))\\
\llbracket\mathbf{exists}(R);\Phi \rrbracket &\leadsto& \lambda e. (\mathsf{not\_empty}~(\llbracket R;\Phi \rrbracket~e))\\
\llbracket \mathbf{binop}(\mathit{op}, v_1, v_2);\Phi\rrbracket &\leadsto&  \lambda e.(\llbracket \mathit{op} \rrbracket~(\llbracket v_1;\Phi \rrbracket~e)~(\llbracket v_2;\Phi \rrbracket~e))\\
\\
 \llbracket v_1,...,v_n;\Phi \rrbracket & \leadsto &\lambda e. (\mathsf{list}~(\llbracket v_1;\Phi\rrbracket~e),...,(\llbracket v_n;\Phi\rrbracket~e)) \\
 \\
\llbracket \mathit{const};\Phi \rrbracket & \leadsto & \lambda e.\mathit{const}\\
\llbracket \mathit{c};\Phi \rrbracket & \leadsto & \lambda e. (\mathsf{lookup}~e~i)~~\textit{where $i=\Phi(c)$}\\
\llbracket \mathbf{aggr}(\alpha, R);\Phi \rrbracket & \leadsto & \lambda e.(\alpha~(\llbracket R;\Phi\rrbracket~e))\\
\\
\mathtt{schema}(T) & = & T.\mathit{schema}\\
\mathtt{schema}(\mathbf{join}(R_1,...,R_n)) & = & \mathtt{schema}( R_1) \concat ...\concat \mathtt{schema}(R_n)\\
\mathtt{schema}(\mathbf{select}(V, R, f))&=& [\mathit{dummy},...,\mathit{dummy}]\\
\mathtt{schema}(\mathbf{rename}(R, \mathit{name}, \bar{c})) &=& [\mathit{name}.c_1,...,\mathit{name}.c_n]\\
\end{array}
\]
\caption{Denotation semantics of $\mathsf{CoreSQL}$, where each $\mathsf{CoreSQL}$ queries is transformed to a Rosette function.}
\label{fig:denotation}
\end{figure}
% \begin{center}
% \AXC{}
% \UIC{$\Sigma \vdash T \leadsto T$}
% \DP
% ~~
% \AXC{$\forall i. \Sigma\vdash Q_i\leadsto T_i$}
% \UIC{$\Sigma \vdash \mathbf{join}(Q_1,...,Q_n) \leadsto T_1\times ... \times T_n$}
% \DP
% ~~
% \AXC{$\Sigma\vdash Q\leadsto T$~~~$\Sigma$}
% \UIC{$\Sigma\vdash \mathbf{select}(\bar{c}, Q, f) \leadsto \lambda t\lambda e. (\mathsf{map}) ~T$}
% \DP
% \end{center}

%!TEX root=writeup.tex
\section{Evaluation}
\label{sec:eval}
To evaluate the effectiveness of our tool in verfication of SQL rewrting rules, we collect 6 different types of SQL rewriting rules as our testing benchmark and evalute our tool on these 6 examples. And our evaluation goal is to answer the following three research quesionts.
\begin{itemize}\itemsep0pt
\item Whether our prototype can verify the correctness of rewriting rules.
\item Whether our prototype can find erros in wrong rewrting rules.
\item How does our prototype scale with the size of the symbolic input tables.
\end{itemize}

\paragraph{Proving Rewriting Rules on Bounded schema}
Code Snippet XXX shows one of our rewriting rules. 
SQL query equivalence with such level of complexity may be hard for a human being to 
figure out, but the verifier can verify the equivalence of queries on a bounded-sized 
schema within a reasnable amount of time.

\begin{figure}
\begin{lstlisting}[style=sql]
SELECT DINSTINCT a, b
FROM R
\end{lstlisting}
\caption{Magic Set Rewrite}
\end{figure}

\paragraph{Detecting Incorrect Rewriting Rules}
Code Snippet XXX is an example of an incorrect SQL rewriting rule.
When running these two SQL queries on our verifier, a counterexample will be returned that
shows in which case the two queries are not equivalent.
This functionality can be potentially useful for SQL query optimization validation.

\paragraph{Scalability}
To illustrate the scalability of our prototype implementation, we evaluated
6 rewriting rules with different symbolic schema sizes on a desktop PC
~\footnote{The PC has 3.60GHz Intel Core i7 CPU, and 8 GB of memeory}.
Figure~\ref{fig:scale} shows the time it takes to verify the equivalence of SQL queries
with different symbolic schema sizes.

\begin{figure}[!htb]
  \centering
  \includegraphics[width=0.7\linewidth]{scale.eps}
  \caption{\# of Symbolic Rows v. Verification Execution Time}
  \label{fig:scale}
\end{figure}

From the figure, we can see that the scalability of the verifier heavily depends 
on the complexity of the rewriting rules.
For those rewriting rules without table join and subquery, it can scales pretty well.
For rewriting rules with either table join or subquery, the prototype implementation can 
not scale well with the growing of the size of the symbolic schema.
Based on our evaluation, queries with both table join and subquery can not scale to schema with
more than 2 rows.

%%% Local Variables:
%%% mode: latex
%%% TeX-master: "writeup"
%%% End:

\section{Teamwork}

All team members contribute to the implementation of this project. The project report is mostly written by Chenglong and Shumo. 
The scalability experiment is done by Kaiyuan.

%%% Local Variables:
%%% mode: latex
%%% TeX-master: "writeup"
%%% End:

\section{Course Topics}

We used the idea of bounded verfication from the course. Here, we 
limit the number of columns and number of symbolic rows an input
table have. We use the solver aided language Rosette covered by 
the course the build our verification tool. The underlying workhorse
for our verification tool is Z3, a SMT solver that also covered by 
the course.

%%% Local Variables:
%%% mode: latex
%%% TeX-master: "writeup"
%%% End:


\bibliographystyle{plain}
\bibliography{writeup}

\end{document}

%%% Local Variables:
%%% mode: latex
%%% TeX-master: t
%%% End:
