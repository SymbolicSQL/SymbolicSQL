\documentclass{article}
\usepackage[pdftex]{graphicx}
\usepackage{wrapfig}
\usepackage{enumerate}
\usepackage{hyperref}
\usepackage{fullpage}
\usepackage{cite}
\usepackage{xcolor}
\usepackage{inconsolata}

\newcommand\note[1]{\textcolor{red}{NOTE: #1}}

\begin{document}

\title{Symbolic Reasoning for SQL Queries}
\author{Chenglong Wang, Kaiyuan Zhang and Shumo Chu}
\date{}
\maketitle

\section*{Abstract}
SQL is a commonly used data query language for database systems. Due to the rich and complex nature of SQL queries, understanding SQL queries is not easy, and this difficulty leads to challenges in the following two scenarios: 1) checking correctness of a SQL optimization rule and 2) debugging a partially corrected SQL. Concretely, these challenges present because SQL is a highly abstract declarative language, and the semantics of a complex SQL query is hard to be inferred directly from its syntax.

To address these challenges, we consider building the formal semantics of SQL in Rosette and employing automatic verification techniques to reason about it. Particularly, we consider the following two applications: 1) bounded checking equivalence of two SQL queries to help verify the correctness of SQL rewriting rules (or provide counter examples when exist), and 2) SQL synthesis based on partially correct queries and counter examples, i.e. help user to correct their wrong queries using counter examples.

\section{Introduction}

\subsection{Verifying query rewriting}
Rewriting SQL query to equivalent queries is an essential part of query
optimizer. Reasoning the correctness of complicated rewriting rigorously 
requires a lot of effort and error prone.
Using the evaluation semantic that we developed, we can build tool to help 
verify the correctness of query rewriting. Given $Q_1$ and $Q_2$, we can use 
the SMT solver to verify the validity of $Q_1 \leftrightarrow Q_2$. 
For complex query rewriting, we may need to bound the size and arity of the
 relation. Then the verification will not be complete but not sound. 
However, we believe that being able to find counter example in bounded case is 
still helpful compared with manually reasoning. 

\section{Overview of This Project}
In this project, we are planning to implement a SQL query equivalence verifier
to validate the equivalence of two SQL queries under certain database schemas.
The final product of this project should be a tool that takes the database schema
definition and two SQL queries, either in the string form or in the parsed AST form, 
as the input, and determine whether these two queries are equivalent under the given
database schema.

As for a general plan, we will first use the similar method as Qex~\cite{veanes2010qex,veanes2009symbolic} did
to formalize the semantic of SQL queries and database schemas.
Then we plan to implement a symbolic execution engine for SQL queries as a prototype 
either by using symbolic engines like Rosette~\cite{torlak2014lightweight} 
or using SMT solvers like z3.
We will try to run some sample queries on the prototype to profile the performance of the
prototype and see how well it can scale.
After that, we may also apply some optimization methods the equivalent verifier 
or even push further by implementing our own theory module for SQL queries for SMT solvers.
\note{This will not happen}



\section{Related Work}

\paragraph{Symbolic SQL Reasoning} Qex~\cite{veanes2010qex,veanes2009symbolic} is a tool to generate tables based on parameterized SQL queries for database system test purpose. Concretely, Qex takes an parameterized SQL query as well as a SQL property as inputs, it will then generate a full SQL query as well as a table instance satisfying the property for database unit test. Qex has build the background theory for SQL queries and is able to translate a SQL query into SMT formulas. Our project uses the same reasoning method as Qex do, but for different application scenario, i.e. SQL semantics checking.

\paragraph{SQL semantics} There exist several previous addressing the formal SQL semantics, including Extended Three Valued Predicate Calculus~\cite{Negri:1991:FSS:111197.111212} and bag semantics~\cite{chinaei2007ordered}. We plan to formal our SQL subset using ordered bag semantics, since this is most commonly used in commercial database systems.

\paragraph{Equivalence of SQL Queries} Chirkova et al.~\cite{chirkova2009equivalence} presents a method to evaluate SQL query equivalence with presence of embedded dependencies. However, their technique addresses only a subset of SQl grammar (conjunctive queries plus aggregation), it is different from our target of equivalence checking for SQL queries with nested subqueries. 

Chakravarthy et al.~\cite{Chakravarthy:1990:LAS:78922.78924} presents a logic based approach for semantics quey optimization. Optimization rules in their systems are generated by the system so that the optimization process is semantics preserving, differently, we are able to bounded-check the whether two SQL optimization rules are equivalent or not, which is more general than their approach..

\bibliography{proposal}{}
\bibliographystyle{plain}

\end{document}
